\section{Introduction}
\subsection{Report layer}
(Voluntary chapter, use if found useful)

\subsection{Contracting entity}
Include necessary information (short description) to give the reader an understanding of the contracting entity and its business. Examples of useful information is history, localization, number of employees, products, business scope, turnover and customers.

\subsection{Problem description}
Describe the contracting entity’s problem description and the background of the problem. It is also important to include external factors that initially affected and limited the choice of solution, in addition to existing applications and networks that relates to the solution.

You may include simple diagrams that illustrates the projects place in the contracting entity’s business. Do not include technical descriptions of details of the final solution in this section.

In design projects, you may think of the result as a “black box”. Your goal is to enable the reader to get a short understanding of what goes into and what comes out of the box, the environment of the box etc. In projects directed against network planning, network implementation and analysis of operational situations, you may have to draft room- and personnel situation in addition to existing data networks etc.

\subsection{Main ide of solution}
Present the contracting entity’s basic idea of what the solution should include (preferably an abstract description). If the contracting entity does not have any concrete ides, this section may transfer to the analysis section. There is also a possibility that he contracting entity has outlined various ideas that require further examination in the analysis section.

\pagebreak