\section{Specification of requirements}
These are the contracting entity’s requirements to how the solution must work, and how to integrate solution with other applications in the business. The requirements must use the contracting entity’s language (i.e., “common” language without too many technical terms that relates directly to the solution), enabling the contracting entity and its users can be confident that you have a correct understanding of the problem. It is important to distinguish between the specification of requirements and the technical documentation, since the latter describes the design and implementation of the final program, network etc. 

In the pre-project, one of the main goals is to discuss the specification of requirements thoroughly with the contracting entity. In some cases, it may be necessary to document such discussions and reviews in the report to illustrate the contracting entity’s motivation and criteria. Changes of the requirements by the contracting entity during the project must reflect in the report. 

The specification of requirements must result in a clearly defined goal for the project. By consulting personnel at the contracting entity and the supervisor at the university, the specification of requirements must define clear goals of final verifiable solutions (prototype with defined limits, tested application, integration and adjustments to existing equipment etc.) that may be completed within the limitations of a bachelor project (workload, theoretical challenges and available resources).

Verifiable solutions indicates that it is possible to evaluate whether the results of the project are according to the specification of requirements. You may advantageously present the specification of requirements pointwise and categorized in a natural manner.

\pagebreak